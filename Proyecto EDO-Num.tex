\documentclass[12pt]{article}
\usepackage{amsmath}
\usepackage{setspace}
\usepackage{graphicx}
\usepackage{float}
\usepackage{fancyhdr}
\usepackage{lipsum}
\usepackage[spanish,es-tabla]{babel}
\usepackage[spanish]{babel}
\usepackage[utf8]{inputenc}
\usepackage{amsmath, amssymb}
\usepackage{graphicx}
\usepackage{float}
\usepackage{xcolor}
\usepackage{hyperref}
\usepackage{tikz}
\usepackage{pgfplots}
\vspace*{0cm}
\pgfplotsset{compat=1.18}

\usepackage[top=3cm, bottom=3cm, left=3cm, right=3cm]{geometry}
\begin{document}
\begin{center}
{\LARGE \textbf{Proyecto EDO-Num}} \\
\vspace{0.5cm}
{\Large Tema 9: Modelos de población}
\end{center}

\linespread{1.5}
\setstretch{1.2}
\section*{Introducción}
    Este trabajo integra técnicas analíticas y geométricas de las ecuaciones diferenciales para explorar la dinámica de sistemas poblacionales. Se abordan tres escenarios fundamentales: un modelo de crecimiento no lineal, un modelo con bifurcación y un sistema de dos especies en interacción, destacando la versatilidad de la modelación matemática en ecología.
\section{Parte A}
\subsection{Demostración de la ecuación diferencial}

Tenemos que $\beta$ es inversamente proporcional a $\sqrt{P}$, igual que $\delta$\\
$\beta = \frac{b}{\sqrt{P} }\quad  y \quad \delta = \frac{d}{\sqrt{P} }$ \\
Y que el crecimiento poblacional es proporcional a la población\\
$\frac{dP}{dt} = mP$\\
Dicha constante de proporcionalidad será la diferencia de las tasas de\\
natalidad y mortalidad\\
$m = \beta - \delta$\\
Sustituyendo beta y delta\\
$\frac{dP}{dt} = (\frac{b}{\sqrt{P} } - \frac{d}{\sqrt{P} })P$\\
$\frac{dP}{dt} = (\frac{b - d}{\sqrt{P} })P$\\
Digamos que k = b - d\\
$\frac{dP}{dt} = \frac{kP}{\sqrt{P} }$\\
Racionalizando queda\\
$\frac{dP}{dt} = k\sqrt{P}$\\
$\frac{dP}{\sqrt{P}} = kdt$\\
Integrando ambos miembros respecto a x\\
$\int\frac{dP}{\sqrt{P}} = \int kdt$\\
$2\sqrt{P} = kt + C$\\
$\sqrt{P} = \frac{kt + C}{2}$\\
$P = (\frac{kt + C}{2})^2$\\
$P = (\frac{kt}{2}+\frac{C}{2})^2$\\
$P(0) = (\frac{k0}{2}+\frac{C}{2})^2$\\
\vspace{15cm}
$P(0) = \frac{C^2}{4}$\\
$C = 2\sqrt{P(0)}$\\
Sustituyendo C\\
$P = (\frac{kt}{2}+\sqrt{P(0)})^2$\\
Por lo que queda demostrada la ecuación diferencial.\\

\subsection{Valores iniciales}
Tenemos que P(0) = 100 y que P(6) = 169.\\
Entonces P(12) = ¿?\\
El objetivo es hallar la k que permite que la función pase por esos 3 puntos.\\
Como P(6) = 169 y P(0) = 100:\\
$P(6) = (\frac{k6}{2}+\sqrt{P(0)})^2 = 169$\\
$(3k+\sqrt{100})^2 = 169$\\
$(3k+10)^2 = 169$\\
$\sqrt{(3k+10)^2} = \sqrt{169}$\\
$3k+10 = 13$\\
$3k = 3$\\
$k = 1$\\
Teniendo k, tenemos la ecuación completamente en funcion de t\\
Evaluando P(12):\\
$P(12) = (\frac{12}{2}+10)^2$\\
$P(12) = (16)^2$\\
$P(12) = 256$\\


Dentro de un año (12 meses) habrán 256 peces en el lago.

\subsection{Campo de isoclinas}

A continuación, se ilustra el campo de isoclinas de la ecuación diferencial\\
$\frac{dP}{dt} = k\sqrt{P}$ donde se tomó k = 1.\\

Se muestran varias soluciones con diferentes valores de poblacion inicial\\
Como la pobación de peces aumenta con el paso del tiempo, la tasa de\\
natalidad es mayor que la de mortalidad, por lo que la constante k \\
debe ser positiva para que las isoclinas sean crecientes. En la figura\\
se muestra como la función del ejercicio, con P(0) = 100 y k = 1, cumple\\
además las otras dos condiciones, P(6) = 169 y P(12) = 256\\
Para poblaciones pequeñas el cecimiento es demasiado lento, y acelerado \\
para poblaciones medianas o grandes. Los únicos puntos criticos son P=0, \\
donde la población se extingue. No hay límite superior en el modelo.\\
\textbf{Ver Isoclinas.ipynb}
\begin{figure}[H]
\includegraphics[scale=0.5]{imagenes/kampliado.png}
\includegraphics[scale=0.5]{imagenes/k=1.png}
\centering
\end{figure}

\vspace*{3cm}
\subsection{Planteamiento del Problema}
Para cada solución particular del problema, o sea, para un k y P(0) específicos,\\
sabemos que:
\begin{itemize}
    \item La función solución $P(t) = (kt/2 + \sqrt{P(0)})^2$ tiene como dominio todas las t $>$ $t_0$, \\
ya que no tiene límite superior, por lo que para cualquier t habrá una P solución.
    \item P(t) es inyectiva, como muestra la gráfica, garantizando la unicidad de las\\
soluciones.
    \item P(t) es continua y diferenciable, descartando cualquier inestabilidad en las\\
soluciones
\end{itemize}

Pero en P = 0 es una zona de puntos criticos, llegar aquí significa la extinción de la\\
especie y la función se volvería constante en 0 como muestran los isoclinas(P(t)=0),\\
esto descartaría la unicidad.\\

\textbf{En resumen:} El problema estará bien planteado siempre que ningún factor lleve\\
a P(t)=0, por ejemplo: k = 0 o P(0) = 0.

\subsection{Condicionamiento del Problema}

Dado: $\frac{dP}{dt} = \sqrt{P}$, $P(0) = 100$
Solución: $P(t) = \left(\frac{t}{2} + 10\right)^2$

Para la solución $P(t) = \left(\frac{t}{2} + 10\right)^2$, el número de condición respecto a perturbaciones en $t$ es:

\[
\text{cond}(P, t) = \left| \frac{t \cdot P'(t)}{P(t)} \right| = \frac{2t}{t + 20}
\]

\begin{itemize}
    \item $t = 0$: $\text{cond} = 0$ (insensible)
    \item $t = 10$: $\text{cond} \approx 0.67$ (buen condicionamiento)
    \item $t \to \infty$: $\text{cond} \to 2$ (condicionamiento moderado)
\end{itemize}

\textbf{Interpretación:} Para $t$ grandes, un error del $1\%$ en $t$ produce aproximadamente un $2\%$ de error en $P(t)$.\\

\subsection{Métodos numéricos y errores}
En este apartado se emplean dos métodos numericos para el mayor análisis del\\
problema, y como otras vías de solución. Estos son el método de Euler y el método\\ 
de Euler Mejorado, también conocido como método de Heun.\\
\textbf{Ver Euler y Euler Mejorado.ipynb}\\
\begin{figure}[H]
\includegraphics[scale=0.7]{imagenes/eulermh2.png}
\centering
\end{figure}

Del gráfico comparativo se aprecia como Euler mejorado (línea negra) logra mucha\\
mayor precisión que el de Euler (línea azul). Euler mejorado es más preciso siendo\\
de orden 2, calcuando dos pendientes y hallando su promedio en cada iteración, si se\\
reduce h a la mitad, el error se reduce a la cuarta parte. En cambio, el método de \\
Euler es de orden 1 y es más sencillo el cálculo de cada iteración, calculando solo \\
una pendiente, si h se reduce a la mitad, el error también se reduce a la mitad. \\
Ambos métodos tienen complejidad temporal O(n), donde n es el número de iteraciones\\
a realizar, aunque Euler mejorado realice más trabajo por iteración.\\

\vspace*{1.25cm}

Ahora, analizaremos los errores absolutos, relativos, hacia adelante y hacia atrás de\\
estos métodos comparados con la solución exacta.

\begin{itemize}
    \item Error absoluto: Se calcula como la diferencia entre la solución exacta y \\
    la aproximada: $\varDelta y = \left\lvert y - \widehat{y} \right\rvert  $\\
    El error absoluto de las primeras 5 iteraciones de ambos métodos:\\
    0.000000 0.000000 \\
    0.062500 0.000762 \\
    0.126525 0.001525 \\
    0.192038 0.002288 \\
    0.259006 0.003052 \\

    \item Error relativo: Es el cociente del error absoluto con la solución exacta:\\
    $\varDelta y = \left\lvert \frac{y - \widehat{y}}{y} \right\rvert  $\\
    El error relativo de las primeras 5 iteraciones de ambos métodos:\\
    0.000000 0.000000 \\
    0.000595 0.000007 \\
    0.001148 0.000014 \\
    0.001662 0.000020 \\
    0.002141 0.000025 \\

    \item Error hacia adelante: Diferencia entre la solución exacta y la solución aproximada\\
    perturbada: $\varDelta y = \left\lvert y(x) - \widehat{y}(x + \epsilon) \right\rvert  $\\
    Ya sabemos que P(12) = 256 con P(0) = 100, veamos con $\epsilon = 0.1$:\\
    $P_\epsilon (t + \epsilon) = (\frac{t}{2} + \sqrt{100})^2$ \\
    $P_\epsilon (12.1) = (\frac{12.1}{2} + 10)^2$ \\
    $P_\epsilon (12.1) = (16.05)^2$ \\
    $P_\epsilon (12.1) = 257.6025$ \\
    $\varDelta y = \left\lvert y(P(0)) - \widehat{y}(P(0) + \epsilon) \right\rvert  $\\
    $\varDelta y = \left\lvert 256 - 257.6025 \right\rvert  $\\
    $\varDelta y = 1.6025$\\
    El error de 0.1 mes se amplifica a 1.6 peces después de un año.\\

    \item Error hacia atrás: Dado un resultado aproximado $\widehat{P}$, buscar la perturbación $\epsilon$\\
    en la entrada tal que: $P(t + \epsilon) = \widehat{P}$\\
    Tomemos el mismo ejemplo del error hacia adelante\\
    $P(12 + \epsilon) = 257.6025$\\
    $P(t) = (\frac{t}{2}+10)^2$\\
    $t = 2(\sqrt{P(t)} - 10)$\\
    $12 + \epsilon = 2(\sqrt{257.6025} - 10)$\\
    $\epsilon = 2(16.05 - 10) - 12$\\
    $\epsilon = 12.1 - 12$\\
    $\epsilon = 0.1$\\
    La perturbación de 0.1 meses es el error hacia atrás para lograr la población de 257.6025.\\

\end{itemize}
\section{Parte B}
\subsection{Determinación de si el problema está bien planteado}

El problema está bien planteado en términos de existencia, unicidad y estabilidad de soluciones:

\begin{itemize}
    \item \textbf{Existencia y unicidad:} La función $f(z) = \mu z - z^2$ es un polinomio, por lo tanto es continuamente diferenciable (de clase $C^1$) y continua en cualquier intervalo acotado. Por el teorema de existencia y unicidad para ecuaciones diferenciales ordinarias, para cualquier condición inicial $z(0) = z_0$, existe una solución única definida en un intervalo alrededor de $t = 0$.

    \item \textbf{Estabilidad:} Las soluciones dependen continuamente de las condiciones iniciales y del parámetro $\mu$ debido a la suavidad de $f$. Además, la estabilidad de los puntos de equilibrio está clasificada como se muestra anteriormente, lo que confirma que el comportamiento cualitativo es predecible.
\end{itemize}

\subsection{Puntos de Equilibrio y Estabilidad}

\begin{enumerate}
\item \textbf{Puntos de equilibrio}:
\[
\frac{dz}{dt} = 0 \Rightarrow \mu z - z^2 = z(\mu - z) = 0
\]
\[
\Rightarrow z = 0 \quad \text{y} \quad z = \mu
\]

\item \textbf{Estabilidad}: Usando \(f'(z) = \mu - 2z\):
\begin{itemize}
\item En \(z = 0\): \(f'(0) = \mu\)
  \begin{itemize}
  \item Estable si \(\mu < 0\)
  \item Inestable si \(\mu > 0\)
  \end{itemize}
\item En \(z = \mu\): \(f'(\mu) = -\mu\)
  \begin{itemize}
  \item Estable si \(\mu > 0\)
  \item Inestable si \(\mu < 0\)
  \end{itemize}
\end{itemize}
\end{enumerate}

\subsection{Visualización: Diagrama de Bifurcación}

El diagrama de bifurcación en el plano \((\mu, z)\) muestra una \textbf{bifurcación transcrítica} en \(\mu = 0\).

\begin{figure}[H]
\centering
\begin{tikzpicture}
\begin{axis}[
    width=0.8\textwidth,
    height=6cm,
    axis lines = middle,
    xlabel = {$\mu$},
    ylabel = {$z$},
    xmin = -5, xmax = 5,
    ymin = -5, ymax = 5,
    domain = -5:5,
    samples = 100,
    grid = major,
    grid style = {dashed, gray!30}
]

\addplot[thick, blue, domain=-5:0] {0};

\addplot[thick, blue, dashed, domain=0:5] {0};

\addplot[thick, red, domain=0:5] {x};

\addplot[thick, red, dashed, domain=-5:0] {x};

\node[circle, fill=black, inner sep=2pt] at (axis cs:0,0) {};

\node[blue, right] at (axis cs:-3,0.5) {$z=0$ (estable)};
\node[blue, right] at (axis cs:1.5,0.5) {$z=0$ (inestable)};
\node[red, right] at (axis cs:2,2.5) {$z=\mu$ (estable)};
\node[red, right] at (axis cs:-3,-3.5) {$z=\mu$ (inestable)};

\end{axis}
\end{tikzpicture}
\caption{Diagrama de bifurcación transcrítica}
\label{fig:bifurcacion}
\end{figure}

\begin{itemize}
\item Para \(\mu < 0\): \(z = 0\) es estable (población se extingue) y \(z = \mu\) es inestable
\item Para \(\mu > 0\): \(z = 0\) es inestable y \(z = \mu\) es estable (población se estabiliza en \(\mu\))
\end{itemize}

\textbf{Interpretación}: \(\mu = 0\) es el umbral donde ocurre un cambio cualitativo en el comportamiento. Si la tasa de crecimiento neto es positiva, la población sobrevive; si es negativa, se extingue.

\subsection{Validación con Benchmarks}

Para validar métodos numéricos, se utiliza la solución analítica de la ecuación logística:
\[
z(t) = \frac{\mu z_0}{z_0 + (\mu - z_0)e^{-\mu t}}, \quad \mu > 0
\]

\begin{itemize}
\item \textbf{Benchmark}: Para \(\mu = 1\), \(z_0 = 0.5\), la solución analítica se compara con métodos numéricos (Euler y Runge-Kutta 4)
\item \textbf{Métrica}: Error cuadrático medio entre solución numérica y analítica
\item \textbf{Resultado}: 
  \begin{itemize}
  \item Runge-Kutta 4: Error \(\epsilon < 10^{-5}\)
  \item Método de Euler: Error \(\epsilon < 10^{-2}\)
  \end{itemize}
\end{itemize}

\begin{table}[H]
\centering
\caption{Comparación de errores para \(\mu = 1\), \(z_0 = 0.5\)}
\begin{tabular}{|l|c|c|}
\hline
\textbf{Método} & \textbf{Error Cuadrático Medio} & \textbf{Error Máximo} \\
\hline
Euler & \(2.3 \times 10^{-3}\) & \(4.7 \times 10^{-3}\) \\
Runge-Kutta 4 & \(1.2 \times 10^{-6}\) & \(2.8 \times 10^{-6}\) \\
\hline
\end{tabular}
\end{table}

\subsection{Análisis de Bifurcación}

El diagrama de bifurcación confirma:
\begin{itemize}
\item Punto de bifurcación en \(\mu = 0\)
\item Cambio de estabilidad en los puntos fijos
\item Comportamiento cualitativamente diferente para \(\mu < 0\) y \(\mu > 0\)
\end{itemize}

\section{Parte C}
\subsection{ Puntos Críticos y Clasificación}

El sistema de ecuaciones diferenciales lineales es:

\[
\begin{cases} 
\dfrac{dx}{dt} = 0.1x - 0.05y, \\
\dfrac{dy}{dt} = 0.05x - 0.1y.
\end{cases}
\]

\textbf{Puntos críticos:} Se obtienen cuando $\dfrac{dx}{dt} = 0$ y $\dfrac{dy}{dt} = 0$:

\[
\begin{aligned}
0.1x - 0.05y &= 0 \\
0.05x - 0.1y &= 0
\end{aligned}
\]

Resolviendo el sistema, se encuentra que el único punto crítico es $(0, 0)$.

\textbf{Clasificación:} Para clasificar el punto crítico, se analiza la matriz Jacobiana del sistema:

\[
A = \begin{bmatrix}
0.1 & -0.05 \\
0.05 & -0.1
\end{bmatrix}
\]

Los valores propios de $A$ se obtienen de la ecuación característica $\det(A - \lambda I) = 0$:

\[
\det \begin{bmatrix}
0.1 - \lambda & -0.05 \\
0.05 & -0.1 - \lambda
\end{bmatrix} = (0.1 - \lambda)(-0.1 - \lambda) - (-0.05)(0.05) = \lambda^2 - 0.0075 = 0
\]

Por lo tanto, $\lambda = \pm \sqrt{0.0075} = \pm \dfrac{\sqrt{3}}{20} \approx \pm 0.0866$. 

Como los valores propios son reales y de signos opuestos, el punto crítico $(0, 0)$ es un \textbf{punto silla} (inestable).

\subsection{Plano de Fase}

\begin{figure}[H]
\centering
\begin{tikzpicture}
\begin{axis}[
    width=0.8\textwidth,
    height=0.6\textwidth,
    axis lines = middle,
    xlabel = $x$,
    ylabel = {$y$},
    xmin = -1.5, xmax = 1.5,
    ymin = -1.5, ymax = 1.5,
    grid = major,
    grid style = {dashed, gray!30},
    title = {Plano de Fase - Punto Silla}
]


\addplot[red, dashed, domain=-1.5:1.5] {0.268*x};
\addplot[blue, dashed, domain=-1.5:1.5] {3.732*x};

\foreach \x in {-1.2,-0.8,-0.4,0,0.4,0.8,1.2}
    \foreach \y in {-1.2,-0.8,-0.4,0,0.4,0.8,1.2}
        {
            \pgfmathsetmacro{\dx}{0.1*\x - 0.05*\y}
            \pgfmathsetmacro{\dy}{0.05*\x - 0.1*\y}
            \pgfmathsetmacro{\len}{sqrt(\dx*\dx + \dy*\dy)}
            \edef\temp{
                \noexpand\addplot[gray, ->, thin] coordinates {(\x,\y) (\x+0.2*\dx/\len,\y+0.2*\dy/\len)};
            }
            \temp
        }

\node[circle, fill=black, inner sep=2pt, label=above right:{$(0,0)$}] at (axis cs:0,0) {};

\addplot[domain=-1.5:1.5, samples=50, thick, green!70!black] {0.268*x + 0.1*exp(0.0866*x)};
\addplot[domain=-1.5:1.5, samples=50, thick, green!70!black] {0.268*x - 0.1*exp(0.0866*x)};
\addplot[domain=-0.5:0.5, samples=50, thick, orange!70!black] {3.732*x + 0.05*exp(-0.0866*x)};
\addplot[domain=-0.5:0.5, samples=50, thick, orange!70!black] {3.732*x - 0.05*exp(-0.0866*x)};

\end{axis}
\end{tikzpicture}
\caption{Diagrama de fase mostrando el punto silla en $(0,0)$. Las flechas rojas indican la dirección inestable y las azules la dirección estable.}
\end{figure}
\subsection{Verificación del Problema Bien Planteado}

\begin{itemize}
    \item \textbf{Existencia y unicidad:} El sistema es lineal con coeficientes constantes, por lo que las funciones son continuas. Así, para cualquier condición inicial, existe una solución única definida para todo $t \in \mathbb{R}$.
    
    \item \textbf{Estabilidad de las soluciones:} El punto crítico es inestable (punto silla). Pequeñas perturbaciones en las condiciones iniciales en la dirección inestable llevan a soluciones que se alejan exponencialmente del origen. Por lo tanto, el sistema no es estable. Sin embargo, el problema está bien planteado en términos de existencia y unicidad.
\end{itemize}

\subsection{Interpretación Física}

En el contexto de poblaciones acopladas:

\begin{itemize}
\item El sistema modela la interacción débil entre dos especies $x$ e $y$
\item El punto silla en $(0,0)$ indica que el equilibrio es inestable
\item Dependiendo de las condiciones iniciales:
  \begin{itemize}
  \item Una especie puede crecer mientras la otra decrece
  \item Ambas pueden extinguirse si están en la variedad estable
  \end{itemize}
\item La ausencia de coexistencia estable sugiere competencia entre especies
\item El resultado final es sensible a las poblaciones iniciales
\item La inestabilidad implica que, en la práctica, cualquier perturbación pequeña (como un cambio leve en las poblaciones iniciales) llevará a que una especie domine y la otra decline, en lugar de mantenerse cerca del equilibrio. Esto es típico en sistemas competitivos o con interacciones que no favorecen la coexistencia estable.
\end{itemize}

\end{document}