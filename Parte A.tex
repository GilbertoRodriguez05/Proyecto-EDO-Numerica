\documentclass[12pt]{article}
\usepackage{amsmath}
\usepackage{setspace}
\usepackage{graphicx}
\usepackage{float}
\usepackage{fancyhdr}
\usepackage{lipsum}
\vspace*{0cm}

\usepackage[top=3cm, bottom=3cm, left=3cm, right=3cm]{geometry}
\begin{document}
\linespread{1.5}
\setstretch{1.2}
\section{Demostración de la ecuación diferencial}

Tenemos que $\beta$ es inversamente proporcional a $\sqrt{P}$, igual que $\delta$\\
$\beta = \frac{b}{\sqrt{P} }\quad  y \quad \delta = \frac{d}{\sqrt{P} }$ \\
Y que el crecimiento poblacional es proporcional a la población\\
$\frac{dP}{dt} = mP$\\
Dicha constante de proporcionalidad será la diferencia de las tasas de\\
natalidad y mortalidad\\
$m = \beta - \delta$\\
Sustituyendo beta y delta\\
$\frac{dP}{dt} = (\frac{b}{\sqrt{P} } - \frac{d}{\sqrt{P} })P$\\
$\frac{dP}{dt} = (\frac{b - d}{\sqrt{P} })P$\\
Digamos que k = b - d\\
$\frac{dP}{dt} = \frac{kP}{\sqrt{P} }$\\
Racionalizando queda\\
$\frac{dP}{dt} = k\sqrt{P}$\\
$\frac{dP}{\sqrt{P}} = kdt$\\
Integrando ambos miembros respecto a x\\
$\int\frac{dP}{\sqrt{P}} = \int kdt$\\
$2\sqrt{P} = kt + C$\\
$\sqrt{P} = \frac{kt + C}{2}$\\
$P = (\frac{kt + C}{2})^2$\\
$P = (\frac{kt}{2}+\frac{C}{2})^2$\\
$P(0) = (\frac{k0}{2}+\frac{C}{2})^2$\\
$P(0) = \frac{C^2}{4}$\\
$C = 2\sqrt{P(0)}$\\
Sustituyendo C\\
$P = (\frac{kt}{2}+\sqrt{P(0)})^2$\\
Por lo que queda demostrada la ecuación diferencial.\\

\vspace*{15cm}
\section{Valores iniciales}
Tenemos que P(0) = 100 y que P(6) = 169.\\
Entonces P(12) = ¿?\\
El objetivo es hallar la k que permite que la función pase por esos 3 puntos.\\
Como P(6) = 169 y P(0) = 100:\\
$P(6) = (\frac{k6}{2}+\sqrt{P(0)})^2 = 169$\\
$(3k+\sqrt{100})^2 = 169$\\
$(3k+10)^2 = 169$\\
$\sqrt{(3k+10)^2} = \sqrt{169}$\\
$3k+10 = 13$\\
$3k = 3$\\
$k = 1$\\
Teniendo k, tenemos la ecuación completamente en funcion de t\\
Evaluando P(12):\\
$P(12) = (\frac{12}{2}+10)^2$\\
$P(12) = (16)^2$\\
$P(12) = 256$\\


Dentro de un año (12 meses) habrán 256 peces en el lago.

\section{Campo de isoclinas}

A continuación, se ilustra el campo de isoclinas de la ecuación diferencial\\
$\frac{dP}{dt} = k\sqrt{P}$ donde se tomó k = 1.\\

Se muestran varias soluciones con diferentes valores de poblacion inicial\\
Como la pobación de peces aumenta con el paso del tiempo, la tasa de\\
natalidad es mayor que la de mortalidad, por lo que la constante k \\
debe ser positiva para que las isoclinas sean crecientes. En la figura\\
se muestra como la función del ejercicio, con P(0) = 100 y k = 1, cumple\\
además las otras dos condiciones, P(6) = 169 y P(12) = 256\\
Para poblaciones pequeñas el cecimiento es demasiado lento, y acelerado \\
para poblaciones medianas o grandes. Los únicos puntos criticos son P=0, \\
donde la población se extingue. No hay límite superior en el modelo.\\
\textbf{Ver Isoclinas.ipynb}
\begin{figure}[H]
\includegraphics[scale=0.5]{imagenes/kampliado.png}
\includegraphics[scale=0.5]{imagenes/k=1.png}
\centering
\end{figure}

\vspace*{3cm}
\section{Planteamiento del Problema}
Para cada solución particular del problema, o sea, para un k y P(0) específicos,\\
sabemos que:
\begin{itemize}
    \item La función solución $P(t) = (kt/2 + \sqrt{P(0)})^2$ tiene como dominio todas las t $>$ $t_0$, \\
ya que no tiene límite superior, por lo que para cualquier t habrá una P solución.
    \item P(t) es inyectiva, como muestra la gráfica, garantizando la unicidad de las\\
soluciones.
    \item P(t) es continua y diferenciable, descartando cualquier inestabilidad en las\\
soluciones
\end{itemize}

Pero en P = 0 es una zona de puntos criticos, llegar aquí significa la extinción de la\\
especie y la función se volvería constante en 0 como muestran los isoclinas(P(t)=0),\\
esto descartaría la unicidad.\\

\textbf{En resumen:} El problema estará bien planteado siempre que ningún factor lleve\\
a P(t)=0, por ejemplo: k = 0 o P(0) = 0.

\section{Condicionamiento del Problema}

Dado: $\frac{dP}{dt} = \sqrt{P}$, $P(0) = 100$
Solución: $P(t) = \left(\frac{t}{2} + 10\right)^2$

Para la solución $P(t) = \left(\frac{t}{2} + 10\right)^2$, el número de condición respecto a perturbaciones en $t$ es:

\[
\text{cond}(P, t) = \left| \frac{t \cdot P'(t)}{P(t)} \right| = \frac{2t}{t + 20}
\]

\begin{itemize}
    \item $t = 0$: $\text{cond} = 0$ (insensible)
    \item $t = 10$: $\text{cond} \approx 0.67$ (buen condicionamiento)
    \item $t \to \infty$: $\text{cond} \to 2$ (condicionamiento moderado)
\end{itemize}

\textbf{Interpretación:} Para $t$ grandes, un error del $1\%$ en $t$ produce aproximadamente un $2\%$ de error en $P(t)$.\\

\section{Métodos numéricos y errores}
En este apartado se emplean dos métodos numericos para el mayor análisis del\\
problema, y como otras vías de solución. Estos son el método de Euler y el método\\ 
de Euler Mejorado, también conocido como método de Heun.\\
\textbf{Ver Euler y Euler Mejorado.ipynb}\\
\begin{figure}[H]
\includegraphics[scale=0.7]{imagenes/eulermh2.png}
\centering
\end{figure}

Del gráfico comparativo se aprecia como Euler mejorado (línea negra) logra mucha\\
mayor precisión que el de Euler (línea azul). Euler mejorado es más preciso siendo\\
de orden 2, calcuando dos pendientes y hallando su promedio en cada iteración, si se\\
reduce h a la mitad, el error se reduce a la cuarta parte. En cambio, el método de \\
Euler es de orden 1 y es más sencillo el cálculo de cada iteración, calculando solo \\
una pendiente, si h se reduce a la mitad, el error también se reduce a la mitad. \\
Ambos métodos tienen complejidad temporal O(n), donde n es el número de iteraciones\\
a realizar, aunque Euler mejorado realice más trabajo por iteración.\\

\vspace*{1.25cm}

Ahora, analizaremos los errores absolutos, relativos, hacia adelante y hacia atrás de\\
estos métodos comparados con la solución exacta.

\begin{itemize}
    \item Error absoluto: Se calcula como la diferencia entre la solución exacta y \\
    la aproximada: $\varDelta y = \left\lvert y - \widehat{y} \right\rvert  $\\
    El error absoluto de las primeras 5 iteraciones de ambos métodos:\\
    0.000000 0.000000 \\
    0.062500 0.000762 \\
    0.126525 0.001525 \\
    0.192038 0.002288 \\
    0.259006 0.003052 \\

    \item Error relativo: Es el cociente del error absoluto con la solución exacta:\\
    $\varDelta y = \left\lvert \frac{y - \widehat{y}}{y} \right\rvert  $\\
    El error relativo de las primeras 5 iteraciones de ambos métodos:\\
    0.000000 0.000000 \\
    0.000595 0.000007 \\
    0.001148 0.000014 \\
    0.001662 0.000020 \\
    0.002141 0.000025 \\

    \item Error hacia adelante: Diferencia entre la solución exacta y la solución aproximada\\
    perturbada: $\varDelta y = \left\lvert y(x) - \widehat{y}(x + \epsilon) \right\rvert  $\\
    Ya sabemos que P(12) = 256 con P(0) = 100, veamos con $\epsilon = 0.1$:\\
    $P_\epsilon (t + \epsilon) = (\frac{t}{2} + \sqrt{100})^2$ \\
    $P_\epsilon (12.1) = (\frac{12.1}{2} + 10)^2$ \\
    $P_\epsilon (12.1) = (16.05)^2$ \\
    $P_\epsilon (12.1) = 257.6025$ \\
    $\varDelta y = \left\lvert y(P(0)) - \widehat{y}(P(0) + \epsilon) \right\rvert  $\\
    $\varDelta y = \left\lvert 256 - 257.6025 \right\rvert  $\\
    $\varDelta y = 1.6025$\\
    El error de 0.1 mes se amplifica a 1.6 peces después de un año.\\

    \item Error hacia atrás: Dado un resultado aproximado $\widehat{P}$, buscar la perturbación $\epsilon$\\
    en la entrada tal que: $P(t + \epsilon) = \widehat{P}$\\
    Tomemos el mismo ejemplo del error hacia adelante\\
    $P(12 + \epsilon) = 257.6025$\\
    $P(t) = (\frac{t}{2}+10)^2$\\
    $t = 2(\sqrt{P(t)} - 10)$\\
    $12 + \epsilon = 2(\sqrt{257.6025} - 10)$\\
    $\epsilon = 2(16.05 - 10) - 12$\\
    $\epsilon = 12.1 - 12$\\
    $\epsilon = 0.1$\\
    La perturbación de 0.1 meses es el error hacia atrás para lograr la población de 257.6025.\\

\end{itemize}

\end{document}